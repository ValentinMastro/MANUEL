\begin{center}
    \begin{tikzpicture}
        \tikzmath{ \r=1.4; }
        \draw (0,0) circle (\r);
        \draw ({\r*cos(30)},{\r*sin(30)}) -- ({\r*cos(210)},{\r*sin(210)});
        \draw ({\r*cos(330)},{\r*sin(330)}) -- ({\r*cos(150)},{\r*sin(150)});
        \draw (0,\r) -- (0,-\r);
    \end{tikzpicture}
\end{center}

\begin{enumerate}
    \item Colorier la roue ci-dessus pour qu'elle soit constituée de deux issues de même probabilité.
    \item Colorier la roue ci-dessus pour que toutes ses issues aient une probabilité de $\frac{1}{3}$.
\end{enumerate}