On donne 12 jetons identiques, puis on les numérote comme ci-dessous et on les range dans une boîte opaque.

\begin{center}
    \begin{tikzpicture}[scale=0.6]
        \foreach[count=\i from 0] \n in {1,3,4,1,5,2,2,5,1,1,3,5} {
            \filldraw[noir] ({3*mod(\i,6)},{-2.5*int(\i/6)}) circle (1);
            \filldraw[white] ({3*mod(\i,6)},{-2.5*int(\i/6)}) node[rouge]{\LARGE\bfseries\n} circle (0.7);
        }
    \end{tikzpicture}
\end{center}

On tire au sort un jeton dans la boîte.
\begin{enumerate}
    \item Définir à l'aide d'une phrase un évènement A dont les issues seraient $\left\{ 1;2;3 \right\}$.
    \item Définir à l'aide d'une phrase un évènement B dont les issues seraient $\left\{ 1;3;5 \right\}$.
    \item Justin affirme qu'il a plus de chances d'obtenir un nombre pair qu'un nombre impair. A-t-il raison ?
\end{enumerate}